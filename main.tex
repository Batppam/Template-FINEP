\documentclass[a4paper,12pt]{article}

%===================Packages===================%
%\usepackage[utf8]{inputenc}
\usepackage[brazil]{babel} % Idioma
\usepackage{fontspec}   % Necessário para fontes no XeLaTeX ou LuaLaTeX
\setmainfont{Arial}     % Define a fonte principal como Arial
\usepackage{changes}
\usepackage{fancyhdr}   % Para cabeçalhos personalizados
\usepackage{xcolor}     % Para trabalhar com cores e opacidade
\usepackage{colortbl}   % Para tabelas coloridas
\usepackage{tabularray} % Tabela com mais recursos
\usepackage{tabularx}
\usepackage{longtable}
\usepackage{transparent}% Para aplicar opacidade a imagens
\usepackage{graphicx}   % Imagens
%\usepackage{geometry}  % Definir margens
\usepackage{mdframed} % Pacote para criar caixas retangulares
\usepackage{gensymb}
\usepackage[a4paper, headsep=2cm, left=1.5cm, right=1.6cm, top=3cm, bottom=4cm]{geometry}
\usepackage{tikz}       % Desenho da borda
\usepackage{array}      % Para ajustar colunas nas tabelas
\usepackage{multirow}   % Para mesclar células na direção vertical
\usepackage[table]{xcolor} % Cores nas tabelas
%\usepackage{everypage} % Para executar comandos no início de cada página
\usepackage{comment}
\usepackage{pdflscape}  % Permite configurar páginas em formato paisagem
%\usepackage{lscape}
\usepackage[table, svgnames, dvipsnames]{xcolor}
\usepackage[table,xcdraw]{xcolor}
\usepackage{makecell, cellspace, caption}
\setlength\cellspacetoplimit{3pt}
\setlength\cellspacebottomlimit{3pt}
\usepackage{multirow}
\usepackage{fancybox}
\usepackage{enumitem}   % Para controle avançado de listas
\usepackage{lipsum}     % Texto de exemplo

\renewcommand{\arraystretch}{1.2} % Aumenta o espaçamento entre as linhas da tabela
%===================Command===================%
%Cabeçalho
\newcommand{\BeneficiariaProponente}{ADD}
\newcommand{\Periodo}{(mm/aa até mm/aa)}
\newcommand{\Titulodoprojeto}{ADD}
\newcommand{\data}{12/01/25}
%-----------------Informações Gerais-----------------%
% 1.1 Beneficiário Proponente
\newcommand{\Razaosocial}{ADD}
\newcommand{\Endereco}{ADD}
\newcommand{\Bairro}{ADD}
\newcommand{\Municipio}{ADD}
%\newcommand{\UF}{ADD}  MUDAR NO TABELA
%\newcommand{\CEP}{ADD} MUDAR NA TABELA
\newcommand{\Telefone}{()}
\newcommand{\Celular}{()}
\newcommand{\email}{ADD}

% 1.2 Projeto
\newcommand{\Titulo}{ADD}
\newcommand{\ObjetivoGeral}{ADD}
\newcommand{\DuracaoPrevista}{ADD}
\newcommand{\ValorTotalFinep}{ADD}
\newcommand{\ValorDesembolsado}{ADD}

% 1.3 Coordenação
\newcommand{\Coordenador}{ADD Coordenador}
\newcommand{\CoordenadorEndereco}{Coordenador  END}
\newcommand{\CoordenadorBairro}{Coordenador BA} 
\newcommand{\CoordenadorMunicipio}{Coordenador MUN}
%\newcommand{\UF}{ADD}  MUDAR NO TABELA
%\newcommand{\CEP}{ADD} MUDAR NA TABELA
\newcommand{\CoordenadorTelefone}{()Coordenador te}
\newcommand{\CoordenadorCelular}{()Coordenador CE}
\newcommand{\Coordenadoremail}{Coordenador EMAIL}

%===================Informações adicionais===================%
% 2.1 Co-executor (caso exista)
\newcommand{\CORazaosocial}{ADD 2}
\newcommand{\COEndereco}{ADD 2}
\newcommand{\COBairro}{ADD 2}
\newcommand{\COMunicipio}{ADD 2}
%\newcommand{\UF}{ADD}  MUDAR NO TABELA
%\newcommand{\CEP}{ADD} MUDAR NA TABELA
\newcommand{\COTelefone}{() 2}
\newcommand{\COCelular}{() 2}
\newcommand{\COemail}{ADD 2}

% 2.1 Co-executor N (caso exista)
\newcommand{\CONRazaosocial}{ADD 2}
\newcommand{\CONEndereco}{ADD 2}
\newcommand{\CONBairro}{ADD 2}
\newcommand{\CONMunicipio}{ADD 2}
%\newcommand{\UF}{ADD}  MUDAR NO TABELA
%\newcommand{\CEP}{ADD} MUDAR NA TABELA
\newcommand{\CONTelefone}{() 2}
\newcommand{\CONCelular}{() 2}
\newcommand{\CONemail}{ADD 2}

% 2.3 Interveniente (caso exista)
\newcommand{\INRazaosocial}{ADD 2}
\newcommand{\INEndereco}{ADIND 2}
\newcommand{\INBairro}{ADD 2}
\newcommand{\INMunicipio}{ADD 2}
%\newcommand{\UF}{ADD}  MUDAR NO TABELA
%\newcommand{\CEP}{ADD} MUDAR NA TABELA
\newcommand{\INTelefone}{() 2}
\newcommand{\INCelular}{() INCelular2}
\newcommand{\INemail}{INemail 2}

% 2.1 Co-executor N (caso exista)
\newcommand{\INNRazaosocial}{ADD 2}
\newcommand{\INNEndereco}{ADD 2}
\newcommand{\INNBairro}{ADD 2}
\newcommand{\INNMunicipio}{ADD 2}
%\newcommand{\UF}{ADD}  MUDAR NO TABELA
%\newcommand{\CEP}{ADD} MUDAR NA TABELA
\newcommand{\INNTelefone}{() 2}
\newcommand{\INNCelular}{() 2}
\newcommand{\INNemail}{ADD 2}

%===================Capa===================%

%\setlength{\headheight}{20pt}% Aumenta a altura do cabeçalho
\setlength{\headsep}{85 pt} % Espaçamento entre o cabeçalho e o texto
% Linhas horizontais no cabeçalho

%===================Rodapé===================%
%\fancyfoot[L]{TESTE}
\fancyfoot[R]{\fontsize{8}{9}\selectfont Versão 1.8.2}
%\fancyfoot[C]{TESTE}
%===================Cabeçalho===================%
% Configuração do cabeçalho
%\renewcommand{\thepage}{} % Remove a numeração de todas as páginas
\pagestyle{fancy}
\renewcommand{\headrulewidth}{0pt} % Remove a linha do cabeçalho

% Cabeçalho à esquerda + imagem opaca
\fancyhead[L]{\vspace{-2.0cm}\hspace{-0.5cm}
\begin{tikzpicture}
    \node[opacity=0.5] {\includegraphics[width=0.18\linewidth]{Imagens/Cabeçalho_img/Logo_cabeca.jpg}};
\end{tikzpicture}}


% Cabeçalho à direita com tabela e texto opacos

 \fancyhead[R]{
    \color[gray]{0.3} % Define opacidade do texto (30% cinza)
    \fontsize{8}{10}\selectfont % Aplica o tamanho de fonte para toda a tabela
    \begin{tabular}{|p{7.28cm}||p{3.59cm}|p{2cm}} % Tabela personalizada
        \multicolumn{2}{l}{\textbf{Relatório técnico de projetos}}   &
        \fontsize{8}{9}\selectfont \data \\  
        \hline\cline{1-2} % Linha horizontal que ocupa toda a largura
    
        \rowcolor{lightgray}
        \textbf{N° do Termo de Outorga} \textit{(01.23.4567.89)}  & 
        \textbf{Referência Finep} \textit{1234/56}  
        \\
    
        \begin{tabular}{c|c|c|c|c|c|c|c|c|c|c|c|c@{\hspace{0.51cm}} c|c|c|c|c|c|c|c|} % Gambiarra
            0 & 
            1 & 
            \cellcolor{lightgray}{\textbf{.}} & 
            2 & 
            3 & 
            \cellcolor{lightgray}{\textbf{.}} & 
            4 & 
            5 & 
            6 & 
            7 & 
            \cellcolor{lightgray}{\textbf{.}} & 
            8 & 
            9 & 
            1 & 
            2 & 
            3 & 
            4 & 
            \cellcolor{lightgray}{\textbf{/}} & 
            5 & 
            6
        \end{tabular} 
        \\\cline{1-2} % Linha que separa as colunas
    
        \rowcolor{lightgray}
        \textbf{Beneficiária Proponente}  &
        \textbf{Período} \textit{(mm/aa até mm/aa)} \\ 
        \BeneficiariaProponente     &                              
        \Periodo  
        \\ \cline{1-2} % Linha que separa as colunas
        \rowcolor{lightgray}
        \multicolumn{2}{|l|} {\textbf{Título do projeto}} \\
        \multicolumn{2}{|l|} {\textbf{ \Titulodoprojeto}} &
        Pág. \thepage /21 % Número da página à direita 
        \\ \hline % Linha que separa as colunas
    \end{tabular}
}

%-------------------------------------Inicio Documento---------------------------------------%

\begin{document}

% Primeira página CAPA com borda
	
    \thispagestyle{empty}
    \newgeometry{top=1cm, bottom=2cm, left=1cm, right=2cm}  % Margens para a capa
\thisfancypage{\setlength{\fboxsep}{15pt}\doublebox}{}

% Adicionar espaço antes da imagem para movê-la para baixo
\vspace*{8cm} 

% Centralizar a imagem
\noindent
\makebox[\textwidth]{%
    \includegraphics[width=0.8\linewidth]{Imagens//Capa_img/Logo_capa.jpg}
}

% Adicionar espaço (se necessário) após a imagem
\vspace{5cm}

\begin{center}
    \Large\textbf{RELATÓRIO TÉCNICO DE PROJETOS \\ DE SUBVENÇÃO ECONÔMICA À INOVAÇÃO}
\end{center}

\vspace{1cm}

\textbf{Termo de Outorga, Nº:} \\

\textbf{Beneficiária Proponente:} \\

\textbf{Período de execução deste relatório \fontsize{8}{10}\selectfont(mm/aa a mm/aa):}

%\restoregeometry % Voltar com as margens
\newpage  
\thispagestyle{empty}

\newgeometry{left=2.5cm, right=2.5cm, top=2.5cm, bottom=2.5cm} % Margens
     
    \tableofcontents  % Sumário

\restoregeometry	
\newpage
	
% Páginas seguintes sem borda
%Informações Gerais
\section{Informações gerais}
\vspace{-0.8cm} % Espaço opcional entre a subseção e a linha
\noindent % Remove o recuo no início da linha
\rule{\textwidth}{2pt} % Linha preenchendo a largura do texto e com 2pt de espessura


%---------------1.1 Beneficiária Proponente---------------%
    
\subsection{Beneficiária Proponente}
\vspace{-1cm} % Subir tabela
%\begin{tabular}{|p{6.5cm}||p{2cm}||p{6cm}|} % Adiciona linhas verticais entre as colunas
\begin{table}[h]
\resizebox{\textwidth}{!}{ % Ajusta a tabela para a largura total da página
\fontsize{8}{10}\selectfont
\begin{tabular}{|p{4.25cm}||p{4.25cm}||p{0.5cm}||p{6cm}|} 
    \multicolumn{3}{l}{\textbf{}}    \\  
    \hline\hline % Linha horizontal que ocupa toda a largura
    \rowcolor{lightgray}
    \multicolumn{3}{|l||}{ Razão social}  & 
    CNPJ \textit{(0123456789/0001-22)}  \\

     \multicolumn{3}{|l||}{\fontsize{8}{10}\selectfont \Razaosocial}  & 
\setlength{\tabcolsep}{2.5pt} % Reduz o espaço entre colunas

\begin{tabular}{c|c|c|c|c|c|c|c|c|c|c|c|c|c|c|c|c|c} 
         0 & 
         1 &  
         2 & 
         3 &  
         4 & 
         5 & 
         6 & 
         7 &  
         8 & 
         9 & 
        \cellcolor{lightgray}{ \textbf{/}} & 
         0 & 
         0 & 
         0 & 
         1 &
        \cellcolor{lightgray}{ \textbf{-}} &
         2 &
         2  
        \end{tabular}
      \\\hline\hline   % Linha que separa as colunas
    \rowcolor{lightgray}
    \multicolumn{3}{|l||}{ {Endereço \textit{(logradouro;n\degree;complemento)}}}  &
     {Bairro} \\ 
    
   \multicolumn{3}{|l||}{ 
   \Endereco}     &                                
   \Bairro 
    \\ \hline\hline % Linha que separa as colunas
    
    \rowcolor{lightgray}
    \multicolumn{2}{|l||}{{ Município}}    &
     UF &
     CEP \textit{(12345-678)} \\
    
    \multicolumn{2}{|l||}{\Municipio}    &
    
    {%\UF
    \setlength{\tabcolsep}{1.5pt}
    \begin{tabular}{c|c}
                       U                   & 
                       F
    \end{tabular}}                                                    &
    
    {%\CEP
    \setlength{\tabcolsep}{7.5pt}
    \begin{tabular}{c|c|c|c|c|c|c|c|c}
                       1                   & 
                       2                   &
                       3                   &
                       4                   &
                       5                   &
        \cellcolor{lightgray}{ \textbf{-}} &
                       6                   &
                       7                   &
                       8                   \\
    \end{tabular}}                                                    
    \\ \hline\hline  % Linha que separa as colunas

    \rowcolor{lightgray}
     Telefone\textit{(cód.de área-n\degree)} &
     Celular\textit{(cód.de área-n\degree)} &
    \multicolumn{2}{l|}{ E-mail} \\
    
    {\Telefone} &
    {\Celular} &
    \multicolumn{2}{l|}{\email}   \\\hline
\end{tabular}}
\end{table}

    %---------------1.2 PROJETO---------------%
\subsection{Projeto}
\vspace{-1cm} % Subir tabela
\begin{table}[h]
\resizebox{\textwidth}{!}{ % Ajusta a tabela para a largura total da página
\fontsize{8}{10}\selectfont
\begin{tabular}{|p{4cm}||p{4cm}||p{3cm}||p{4cm}|} 
    \multicolumn{4}{l}{\textbf{}}    \\ \hline\hline % Linha horizontal que ocupa toda a largura
    \rowcolor{lightgray}
    \multicolumn{4}{|l|}{ Título}   \\
    \multicolumn{4}{|l|}{ \Titulo}   \\ \hline\hline

    \rowcolor{lightgray}
    \multicolumn{4}{|l|}{ Objetivo Geral} \\
    \multicolumn{4}{|l|}{ \ObjetivoGeral}  \\ \hline\hline

    \rowcolor{lightgray}
     Data de Assinatura (dd/mm/aa) &
     Duração prevista(meses)       &
     Valor Total-Finep(R\$)        &
     Valor desembolsado-Finep(R\$) \\

    \setlength{\tabcolsep}{5.3pt}
    \begin{tabular}{c|c|c|c|c|c|c|c}

%---------------MUDAR DATA-----------------%
                      2                    & 
                      0                    &
        \cellcolor{lightgray}{\textbf{/}}  & 
                      1                    &
                      2                    &
        \cellcolor{lightgray}{\textbf{/}} & 
                      2                    &
                      5                    
    \end{tabular} &
    
     \DuracaoPrevista              &
     \ValorTotalFinep              &
     \ValorDesembolsado         \\\hline
\end{tabular}}
\end{table}

%---------------1.3 Coordenação---------------%
\subsection{Coordenação}
\vspace{-1.2cm} % Subir tabela
%\begin{tabular}{|p{6.5cm}||p{2cm}||p{6cm}|} % Adiciona linhas verticais entre as colunas
\begin{table}[h]
\resizebox{\textwidth}{!}{ % Ajusta a tabela para a largura total da página
\fontsize{8}{10}\selectfont
\begin{tabular}{|p{4.25cm}||p{4.25cm}||p{0.5cm}||p{6cm}|} 
    \multicolumn{3}{l}{\textbf{}}    \\  
    \hline\hline % Linha horizontal que ocupa toda a largura
    \rowcolor{lightgray}
    \multicolumn{3}{|l||}{Coordenador \textit{(nome completo sem abreviações)}}  & 
     CNPJ \textit{(0123456789/0001-22)}  \\

     \multicolumn{3}{|l||}{\Coordenador}  & 
\setlength{\tabcolsep}{2.5pt} % Reduz o espaço entre colunas
\begin{tabular}{c|c|c|c|c|c|c|c|c|c|c|c|c|c|c|c|c|c} 
         0 & 
         1 &  
         2 & 
         3 &  
         4 & 
         5 & 
         6 & 
         7 &  
         8 & 
         9 & 
        \cellcolor{lightgray}{ \textbf{/}} & 
         0 & 
         0 & 
         0 & 
         1 &   
        \cellcolor{lightgray}{ \textbf{-}} &
         2 &
         2  
        \end{tabular}
      \\\hline\hline   % Linha que separa as colunas
    \rowcolor{lightgray}
    \multicolumn{3}{|l||}{ {Endereço \textit{(logradouro;n\degree;complemento)}}}  &
    {Bairro} \\ 
    
   \multicolumn{3}{|l||}{  \CoordenadorEndereco}     &                              
       \CoordenadorBairro 
    \\ \hline\hline % Linha que separa as colunas
    
    \rowcolor{lightgray}
    \multicolumn{2}{|l||}{{ Município}}    &
     UF &
     CEP \textit{(12345-678)} \\
    
    \multicolumn{2}{|l||}{  \CoordenadorMunicipio}    &
    
    {%\UF
    \setlength{\tabcolsep}{1.5pt}
    \begin{tabular}{c|c}
                       U                   & 
                       F
    \end{tabular}}                                                    &
    
    {%\CEP
    \setlength{\tabcolsep}{7.5pt}
    \begin{tabular}{c|c|c|c|c|c|c|c|c}
                       1                   & 
                       2                   &
                       3                   &
                       4                   &
                       5                   &
        \cellcolor{lightgray}{ \textbf{-}} &
                       6                   &
                       7                   &
                       8                   \\
    \end{tabular}}                                                    
    \\ \hline\hline  % Linha que separa as colunas

    \rowcolor{lightgray}
    \fontsize{8}{10}\selectfont Telefone\textit{(cód.de área-n\degree)} &
    \fontsize{8}{10}\selectfont Celular\textit{(cód.de área-n\degree)} &
    \multicolumn{2}{l|}{\fontsize{8}{10}\selectfont E-mail} \\
    
    {\fontsize{8}{10}\selectfont  \CoordenadorTelefone} &
    {\fontsize{8}{10}\selectfont  \CoordenadorCelular} &
    \multicolumn{2}{l|}{\fontsize{8}{10}\selectfont  \Coordenadoremail}   \\\hline
\end{tabular}}
\end{table}

%-------------Informações Adicionais-------------%
\newpage

\section{Informações adicionais}
\vspace{-0.8cm} % Espaço opcional entre a subseção e a linha
\noindent % Remove o recuo no início da linha
\rule{\textwidth}{2pt} % Linha preenchendo a largura do texto e com 2pt de espessura

%Co-executor 
\subsection{Co-executor (caso exista)}
\vspace{-0.7cm} % Subir tabela
%\begin{tabular}{|p{6.5cm}||p{2cm}||p{6cm}|} % Adiciona linhas verticais entre as colunas
\begin{table}[h]
\resizebox{\textwidth}{!}{ % Ajusta a tabela para a largura total da página
\fontsize{8}{10}\selectfont
\begin{tabular}{|p{4.25cm}||p{4.25cm}||p{0.5cm}||p{6cm}|} 
    \multicolumn{3}{l}{\textbf{}}    \\  
    \hline\hline % Linha horizontal que ocupa toda a largura
    \rowcolor{lightgray}
    \multicolumn{3}{|l||}{ Razão Social}  & 
                            CNPJ \textit{(0123456789/0001-22)}  \\

     \multicolumn{3}{|l||}{\fontsize{8}{10}\selectfont \Razaosocial2}  & 
\setlength{\tabcolsep}{2.599pt} % Reduz o espaço entre colunas
\begin{tabular}{c|c|c|c|c|c|c|c|c|c|c|c|c|c|c|c|c|c} 
         0 & 
         1 &  
         2 & 
         3 &  
         4 & 
         5 & 
         6 & 
         7 &  
         8 & 
         9 & 
        \cellcolor{lightgray}{\textbf{/}} & 
         0 & 
         0 & 
         0 & 
         1 &
        \cellcolor{lightgray}{\textbf{-}} &
         2 &
         2  
        \end{tabular}
      \\\hline\hline   % Linha que separa as colunas
    \rowcolor{lightgray}
    \multicolumn{3}{|l||}{{Endereço \textit{(logradouro;n\degree;complemento)}}}  &
     {Bairro} \\ 

       \multicolumn{3}{|l||}{   \COEndereco}     &                              
       \COBairro 
    \\ \hline\hline % Linha que separa as colunas
    
    \rowcolor{lightgray}
    \multicolumn{2}{|l||}{{\ Município}}    &
     UF &
     CEP \textit{(12345-678)} \\
    
    \multicolumn{2}{|l||}{  \COMunicipio}  &
    
    {%===================\UF===================
    \setlength{\tabcolsep}{1.5pt}
    \begin{tabular}{c|c}
                       U                   & 
                       F
    \end{tabular}}                         &
    
    {%===================\CEP===================
    \setlength{\tabcolsep}{7.5pt}
    \begin{tabular}{c|c|c|c|c|c|c|c|c}
                       1                   & 
                       2                   &
                       3                   &
                       4                   &
                       5                   &
        \cellcolor{lightgray}{ \textbf{-}} &
                       6                   &
                       7                   &
                       8                   \\
    \end{tabular}}                                                    
    \\ \hline\hline  % Linha que separa as colunas

    \rowcolor{lightgray}
     Telefone\textit{(cód.de área-n\degree)} &
     Celular\textit{(cód.de área-n\degree)} &
    \multicolumn{2}{l|}{ E-mail} \\
    
    {  \COTelefone} &
    {  \COCelular} &
    \multicolumn{2}{l|}{  \COemail}   \\\hline
\end{tabular}}
\end{table}
%---------------------------Co-executor N -------------------------%

\subsection{Co-executor N (caso exista)}
\vspace{-0.7cm} % Subir tabela
\begin{table}[h]  
\resizebox{\textwidth}{!}{ % Ajusta a tabela para a largura total da página
\fontsize{8}{10}\selectfont
\begin{tabular}{|p{4.25cm}||p{4.25cm}||p{0.5cm}||p{6cm}|} 
    \multicolumn{3}{l}{\textbf{}}    \\  
    \hline\hline % Linha horizontal que ocupa toda a largura
    \rowcolor{lightgray}
    \multicolumn{3}{|l||}{ Razão Social}     & 
    \fontsize{8}{10}\selectfont CNPJ \textit{(0123456789/0001-22)}  \\

     \multicolumn{3}{|l||}{ \CONRazaosocial} & 
\setlength{\tabcolsep}{2.5pt} % Reduz o espaço entre colunas
\begin{tabular}{c|c|c|c|c|c|c|c|c|c|c|c|c|c|c|c|c|c} 
         0 & 
         1 &  
         2 & 
         3 &  
         4 & 
         5 & 
         6 & 
         7 &  
         8 & 
         9 & 
        \cellcolor{lightgray}{\textbf{/}} & 
         0 & 
         0 & 
         0 & 
         1 &
        \cellcolor{lightgray}{\textbf{-}} &
         2 &
         2  
        \end{tabular}
      \\\hline\hline   % Linha que separa as colunas
    \rowcolor{lightgray}
    \multicolumn{3}{|l||}{{Endereço \textit{(logradouro;n\degree;complemento)}}}  &   {Bairro} \\ 
    
   \multicolumn{3}{|l||}{\CONEndereco}     &        \CONBairro 
    \\ \hline\hline % Linha que separa as colunas
    
    \rowcolor{lightgray}
    \multicolumn{2}{|l||}{{Município}}       &
    UF &
    CEP \textit{(12345-678)} \\
    
    \multicolumn{2}{|l||}{\CONMunicipio}    &
    
    {%\UF
    \setlength{\tabcolsep}{1.5pt}
    \begin{tabular}{c|c}
                       U                   & 
                       F
    \end{tabular}}                        &
    
    {%\CEP
    \setlength{\tabcolsep}{7.5pt}
    \begin{tabular}{c|c|c|c|c|c|c|c|c}
                       1                   & 
                       2                   &
                       3                   &
                       4                   &
                       5                   &
        \cellcolor{lightgray}{\fontsize{8}{10}\selectfont \textbf{-}} &
                       6                   &
                       7                   &
                       8                   \\
    \end{tabular}}                                                    
    \\ \hline\hline  % Linha que separa as colunas

    \rowcolor{lightgray}
     Telefone\textit{(cód.de área-n\degree)} & %Telefone
     Celular\textit{(cód.de área-n\degree)} &  %Celular
    \multicolumn{2}{l|}{ E-mail} \\            %Email
    
    {\CONTelefone} &
    {\CONCelular} &
    \multicolumn{2}{l|}{\CONemail}   \\\hline
\end{tabular}}
\end{table}
\newpage
%---------------------------INTERVENIENTE-------------------------%

%Co-executor 
\subsection{Interveniente (caso exista)}
\vspace{-0.7cm} % Subir tabela
%\begin{tabular}{|p{6.5cm}||p{2cm}||p{6cm}|} % Adiciona linhas verticais entre as colunas
\begin{table}[h]
\resizebox{\textwidth}{!}{ % Ajusta a tabela para a largura total da página
\fontsize{8}{10}\selectfont
\begin{tabular}{|p{4.25cm}||p{4.25cm}||p{0.5cm}||p{6cm}|} 
    \multicolumn{3}{l}{\textbf{}}    \\  
    \hline\hline % Linha horizontal que ocupa toda a largura
    \rowcolor{lightgray}
    \multicolumn{3}{|l||}{ Razão Social}  & 
                            CNPJ \textit{(0123456789/0001-22)}  \\

     \multicolumn{3}{|l||}{\fontsize{8}{10}\selectfont \INRazaosocial}  & 
\setlength{\tabcolsep}{2.599pt} % Reduz o espaço entre colunas
\begin{tabular}{c|c|c|c|c|c|c|c|c|c|c|c|c|c|c|c|c|c} 
         0 & 
         1 &  
         2 & 
         3 &  
         4 & 
         5 & 
         6 & 
         7 &  
         8 & 
         9 & 
        \cellcolor{lightgray}{\textbf{/}} & 
         0 & 
         0 & 
         0 & 
         1 &
        \cellcolor{lightgray}{\textbf{-}} &
         2 &
         2  
        \end{tabular}
      \\\hline\hline   % Linha que separa as colunas
    \rowcolor{lightgray}
    \multicolumn{3}{|l||}{{Endereço \textit{(logradouro;n\degree;complemento)}}}  &
     {Bairro} \\ 

       \multicolumn{3}{|l||}{   \INEndereco}     &                              
       \COBairro 
    \\ \hline\hline % Linha que separa as colunas
    
    \rowcolor{lightgray}
    \multicolumn{2}{|l||}{{\ Município}}    &
     UF &
     CEP \textit{(12345-678)} \\
    
    \multicolumn{2}{|l||}{  \INMunicipio}  &
    
    {%===================\UF===================
    \setlength{\tabcolsep}{1.5pt}
    \begin{tabular}{c|c}
                       U                   & 
                       F
    \end{tabular}}                         &
    
    {%===================\CEP===================
    \setlength{\tabcolsep}{7.5pt}
    \begin{tabular}{c|c|c|c|c|c|c|c|c}
                       1                   & 
                       2                   &
                       3                   &
                       4                   &
                       5                   &
        \cellcolor{lightgray}{ \textbf{-}} &
                       6                   &
                       7                   &
                       8                   \\
    \end{tabular}}                                                    
    \\ \hline\hline  % Linha que separa as colunas

    \rowcolor{lightgray}
     Telefone\textit{(cód.de área-n\degree)} &
     Celular\textit{(cód.de área-n\degree)} &
    \multicolumn{2}{l|}{ E-mail} \\
    
    {  \INTelefone} &
    {  \INCelular} &
    \multicolumn{2}{l|}{  \INemail}   \\\hline
\end{tabular}}
\end{table}
%---------------------------Interveniente N -------------------------%

\subsection{Interveniente N (caso exista)}
\vspace{-0.7cm} % Subir tabela
\begin{table}[h]  
\resizebox{\textwidth}{!}{ % Ajusta a tabela para a largura total da página
\fontsize{8}{10}\selectfont
\begin{tabular}{|p{4.25cm}||p{4.25cm}||p{0.5cm}||p{6cm}|} 
    \multicolumn{3}{l}{\textbf{}}    \\  
    \hline\hline % Linha horizontal que ocupa toda a largura
    \rowcolor{lightgray}
    \multicolumn{3}{|l||}{ Razão Social}     & 
    \fontsize{8}{10}\selectfont CNPJ \textit{(0123456789/0001-22)}  \\

     \multicolumn{3}{|l||}{ \INNRazaosocial} & 
\setlength{\tabcolsep}{2.5pt} % Reduz o espaço entre colunas
\begin{tabular}{c|c|c|c|c|c|c|c|c|c|c|c|c|c|c|c|c|c} 
         0 & 
         1 &  
         2 & 
         3 &  
         4 & 
         5 & 
         6 & 
         7 &  
         8 & 
         9 & 
        \cellcolor{lightgray}{\textbf{/}} & 
         0 & 
         0 & 
         0 & 
         1 &
        \cellcolor{lightgray}{\textbf{-}} &
         2 &
         2  
        \end{tabular}
      \\\hline\hline   % Linha que separa as colunas
    \rowcolor{lightgray}
    \multicolumn{3}{|l||}{{Endereço \textit{(logradouro;n\degree;complemento)}}}  &   {Bairro} \\ 
    
   \multicolumn{3}{|l||}{\INNEndereco}     &        \INNBairro 
    \\ \hline\hline % Linha que separa as colunas
    
    \rowcolor{lightgray}
    \multicolumn{2}{|l||}{{Município}}       &
    UF &
    CEP \textit{(12345-678)} \\
    
    \multicolumn{2}{|l||}{\INNMunicipio}    &
    
    {%\UF
    \setlength{\tabcolsep}{1.5pt}
    \begin{tabular}{c|c}
                       U                   & 
                       F
    \end{tabular}}                        &
    
    {%\CEP
    \setlength{\tabcolsep}{7.5pt}
    \begin{tabular}{c|c|c|c|c|c|c|c|c}
                       1                   & 
                       2                   &
                       3                   &
                       4                   &
                       5                   &
        \cellcolor{lightgray}{\textbf{-}} &
                       6                   &
                       7                   &
                       8                   \\
    \end{tabular}}                                                    
    \\ \hline\hline  % Linha que separa as colunas

    \rowcolor{lightgray}
     Telefone\textit{(cód.de área-n\degree)} & %Telefone
     Celular\textit{(cód.de área-n\degree)} &  %Celular
    \multicolumn{2}{l|}{ E-mail} \\            %Email
    
    {\INNTelefone} &
    {\INNCelular} &
    \multicolumn{2}{l|}{\INNemail}   \\\hline
\end{tabular}}
\end{table}

\newpage

%=============== 3 Tabela resumo da execução do cronograma físico do projeto ==============%
\begin{landscape}
%\newgeometry{left=1cm, right=1cm, top=5cm, bottom=2.5cm} % Margens
    \thispagestyle{empty}
    \section{Tabela resumo da execução do cronograma físico do projeto\protect\footnote[1]{Veja instruções de preenchimento no final do relatório.}}
    \vspace{-0.8cm} % Espaço opcional entre a subseção e a linha
    \rule{\textwidth}{2pt} % Linha preenchendo a largura do texto e com 2pt de espessura

  \begin{table}[h]
    \fontsize{8}{10}\selectfont
    \centering
    \resizebox{\textwidth}{!}{ % Ajusta a tabela para a largura total da página
    \begin{tabular}{|c|c|c|c|c|c|c|c|}
        \hline
        \rowcolor{lightgray}
        \multicolumn{2}{|c|}{\textbf{Duração prevista para o projeto}} & 
        \multicolumn{2}{c|}{\textbf{Duração efetiva do projeto}} & 
        \multicolumn{2}{c|}{\textbf{Percentual do projeto executado no período}} & 
        \multicolumn{2}{c|}{\textbf{Percentual acumulado do projeto}} \\ \hline
        
        \rowcolor{lightgray}
        \textbf{Mês/Ano Início} & \textbf{Mês/Ano Fim} & 
        \textbf{Mês/Ano Início} & \textbf{Mês/Ano Fim} & 
        \textbf{Previsto (\%)} & \textbf{Realizado (\%)} & 
        \textbf{Previsto (\%)} & \textbf{Realizado (\%)} \\ \hline
        
        início & fim & início & fim & (\%) & (\%) & (\%) & (\%) \\ \hline
    \end{tabular}}
\end{table}

\begin{table}[h]
\centering
\fontsize{12}{10}\selectfont
\resizebox{\textwidth}{!}{ % Ajusta a tabela para a largura total da página
\begin{tabular}{|p{2cm}|p{3cm}|p{4cm}|p{3cm}|p{2cm}|p{2cm}|p{2cm}|p{2cm}|p{2cm}|p{2cm}|p{2cm}|p{2cm}|}
    \hline
    \hline
    \rowcolor{lightgray}
    \textbf{Item Meta}&
    \textbf{Metas}&
    \textbf{Atividades}&
    \textbf{Indicador \newline Físico}&
    \multicolumn{2}{c|}{\textbf{Duração Prevista}} &
    \multicolumn{2}{c|}{\textbf{Duração efetiva}} &
    \multicolumn{2}{c|}{\textbf{Executado no período}} &
    \multicolumn{2}{c|}{\parbox[t]{3cm}{\textbf{Acumulado da meta atividade}}}  \\ \cline{5-12}
    
    \rowcolor{lightgray}
    & & & & 
    \textbf{Mês/Ano inicio} & \textbf{Mês/Ano fim} &
    \textbf{Mês/Ano inicio} & \textbf{Mês/Ano fim} &
    \textbf{Previsto (\%)} & \textbf{Realizado (\%)} & \textbf{Previsto (\%)} & \textbf{Realizado (\%)} 
    \\ \hline


    %PREENCHER [1]
    [1] & 
    Texto descrevendo a meta física 1 
    &
    &
    & 01/10 
    & 12/10 
    & 02/10 
    & - & 
    6m/12m = 50\% & 
    4m/12m = 33\% & 
    6m/12m = 50\% &
    4m/12m = 33\% \\
    \hline

   %PREENCHER [1.1]
    [1.1] & 
    Texto descrevendo a meta física 1 
    &
    &
    & 01/10 
    & 12/10 
    & 02/10 
    & - & 
    6m/12m = 50\% & 
    4m/12m = 33\% & 
    6m/12m = 50\% &
    4m/12m = 33\% \\
    \hline 
\end{tabular}}\end{table}
 


\end{landscape}
\restoregeometry % Voltar com as margens
\newpage

%=================== 4.Execução do cronograma físico do projeto ===================%

\section{Execução do cronograma físico do projeto\protect\footnote[2]{Veja instruções de preenchimento no final do relatório.}.}
\vspace{-0.8cm} % Espaço opcional entre a subseção e a linha
    \rule{\textwidth}{2pt} % Linha preenchendo a largura do texto e com 2pt de espessura
\subsection{Metas físicas\protect\footnote[3]{Todas as metas e atividades associadas que constam do cronograma físico, aprovado pela Finep para o projeto, devem ser mencionadas no relatório de acompanhamento técnico.}.}
%\vspace{-0.7cm} % Subir tabela

% META FÍSICA 1
\begin{table}[h]
\fontsize{8}{10}\selectfont
\centering
\begin{tabular}{| p{9cm} | p{1.5cm} | p{1.5cm} | p{1.5cm} | p{1.5cm} |}
    \hline
    \rowcolor{lightgray}
    \multirow{3}{*}{\textbf{Meta física 1}}&
    \multicolumn{2}{c|}{\textbf{Executado no período}} &
    \multicolumn{2}{c|}{\parbox[t]{3cm}{\textbf{Acumulado da meta \newline durante o projeto}}}  \\ \cline{2-5}
    
    \rowcolor{lightgray}
    & \textbf{Previsto (\%)} & \textbf{Realizado (\%)} & \textbf{Previsto (\%)} & \textbf{Realizado (\%)} \\ \hline

    [Texto descrevendo a meta física] & 50 & 33 & 50 & 33 \\ \hline
\end{tabular}
\\[0.5cm] % Espaço entre as tabelas

% ATIVIDADE 1.1
\begin{tabular}{| p{4.05cm} | p{4.5cm} | p{1.5cm} | p{1.5cm} | p{1.5cm} | p{1.5cm} |}
    \hline
    \rowcolor{lightgray}
    \multirow{2}{*}{\textbf{Atividade 1.1}} & 
    \multirow{2}{*}{\textbf{Indicador físico de execução}} & 
    \multicolumn{2}{c|}{\textbf{Executado no período}} &
    \multicolumn{2}{c|}{\parbox[t]{3cm}{\textbf{Acumulado da meta \newline durante o projeto}}}  \\ \cline{3-6}
    
    \rowcolor{lightgray}
    & & \textbf{Previsto (\%)} & \textbf{Realizado (\%)} & \textbf{Previsto (\%)} & \textbf{Realizado (\%)} \\ \hline

    [Texto descrevendo a atividade] & [Texto descrevendo o indicador físico] & 75 & 50 & 75 & 50 \\ \hline
\end{tabular}
\\[0.5cm] % Espaço entre as tabelas

% ATIVIDADE 1.2
\begin{tabular}{| p{4.05cm} | p{4.5cm} | p{1.5cm} | p{1.5cm} | p{1.5cm} | p{1.5cm} |}
    \hline
    \rowcolor{lightgray}
    \multirow{2}{*}{\textbf{Atividade 1.1}} & 
    \multirow{2}{*}{\textbf{Indicador físico de execução}} & 
    \multicolumn{2}{c|}{\textbf{Executado no período}} &
    \multicolumn{2}{c|}{\parbox[t]{3cm}{\textbf{Acumulado da meta \newline durante o projeto}}}  \\ \cline{3-5}
    
    \rowcolor{lightgray}
    & & \textbf{Previsto (\%)} & \textbf{Realizado (\%)} & \textbf{Previsto (\%)} & \textbf{Realizado (\%)} \\ \hline

    [Texto descrevendo a atividade] & [Texto descrevendo o indicador físico] & 75 & 50 & 75 & 50 \\ \hline %MUDAR VALORES
\end{tabular}
\end{table}

\begin{table}[h]
\begin{longtable}{|p{0.94\textwidth}|} % Ajuste a largura para 90% da largura da página
    \hline
    \rowcolor{lightgray} 
    \fontsize{8}{10}\selectfont Descreva o desenvolvimento da atividade \\ \hline 
    \endhead
    
    \fontsize{10}{12}\selectfont 
    TEXTO \\ \hline
    
    \rowcolor{lightgray} 
    \fontsize{8}{10}\selectfont Comente sobre o(s) resultado(s) \textit{(em caso de tarefa concluída, o indicador físico deverá constar como anexo no relatório)} \\
    \hline
    {\fontsize{10}{12}\selectfont TEXTO} \\ \hline
    
    \rowcolor{lightgray} 
    \fontsize{8}{10}\selectfont Justifique o eventual atraso ou adiantamento da execução da tarefa, em relação à previsão inicial \\
    \hline
    {\fontsize{10}{12}\selectfont TEXTO} \\ \hline
\end{longtable}
\end{table}



\newpage


%===================5 - Avaliação da gestão do projeto===================%
\section{Avaliação da gestão do projeto}
\vspace{-0.8cm} % Espaço opcional entre a linha
\rule{\textwidth}{2pt} % Linha preenchendo a largura do texto e com 2pt de espessura
\vspace{-0.9cm} % Subir tabela
\begin{table}[h]
    \begin{longtable}{|p{17.4cm}|} % Definindo a largura da coluna
     \hline
     \rowcolor{lightgray} 
     \fontsize{8}{10}\selectfont{Apresente a(s) alteração(ões) na equipe executora (em caso de inclusão ou substituição de algum membro na equipe, deverão ser enviados os comprovantes da formação/titulação, e informados salário e número de horas dedicadas ao projeto\protect\footnote[4]{Toda alteração na equipe executora, pretendida pela proponente, deve ser autorizada previamente pela Finep. Portanto, não basta apresentá-la no relatório, sem que tenha sido analisada e previamente autorizada.}}  
     \\ \hline \endhead
     
     \fontsize{10}{12}\selectfont 
     
     
     TEXTO
     
     
     
     \\ \hline
     \end{longtable}

    
     \begin{longtable}{|p{17.4cm}|} % Definindo a largura da coluna
     \rowcolor{lightgray} 
     \fontsize{8}{10}\selectfont{Relacione e associe às atividades do projeto eventual(ais) capacitação(ões) adicional(ais) adquirida(s) ou gerada(s) pela equipe executora, em função do desenvolvimento do projeto.} \\\hline
     \fontsize{10}{10}\selectfont 
     
     
     TEXTO \\
 
     
     \hline
\end{longtable}

\begin{longtable}{|p{17.4cm}|} % Definindo a largura da coluna
     \rowcolor{lightgray} 
     \fontsize{8}{10}\selectfont{Mencione eventual(ais) melhoria(s) nas instalações físicas proporcionadas pelo projeto} \\\hline
     \fontsize{10}{12}\selectfont 
     
     
     TEXTO
     
     
     \\ \hline
\end{longtable}

\begin{longtable}{|p{17.4cm}|} % Definindo a largura da coluna
     \rowcolor{lightgray} 
     \fontsize{10}{10}\selectfont{Relacione eventual(ais) dificuldade(s) não-técnicas do projeto (administrativas, financeiras, etc)\footnote[5]{Este campo não se destina às solicitações de remanejamento financeiro, as quais deverão ser analisadas e autorizadas previamente pela Finep.}} \\\hline
     \fontsize{10}{12}\selectfont 
     
     
     TEXTO
     
     
     \\ \hline
\end{longtable}
\end{table}


\newpage

%===================6 - Impactos internos e externos do projeto===================%
\section{Impactos internos e externos do projeto}
\vspace{-0.8cm} % Espaço opcional entre a linha
\rule{\textwidth}{2pt} % Linha preenchendo a largura do texto e com 2pt de espessura
\vspace{-0.9cm} % Subir tabela
\begin{table}[h]
    \begin{longtable}{|p{17.4cm}|} % Definindo a largura da coluna
     \hline
     \rowcolor{lightgray} 
     \fontsize{8}{10}\selectfont{Mencione as perspectivas de desdobramentos que o projeto proporcionou às atividades internas da instituição executora e/ou parceiros, incluindo mudanças organizacionais, de patamar de faturamento, etc.}  
     \\ \hline \endhead
     
     \fontsize{10}{12}\selectfont 
     
     
     TEXTO
     
     
     
     \\ \hline
     \end{longtable}

    
     \begin{longtable}{|p{17.4cm}|} % Definindo a largura da coluna
     \rowcolor{lightgray} 
     \fontsize{8}{10}\selectfont{Descreva eventual (is) mudança (s) do posicionamento da empresa perante o mercado, proporcionada (s) pelo projeto} \\\hline
     \fontsize{10}{10}\selectfont 
     
     
     TEXTO \\
 
     
     \hline
\end{longtable}

\begin{longtable}{|p{17.4cm}|} % Definindo a largura da coluna
     \rowcolor{lightgray} 
     \fontsize{8}{10}\selectfont{Apresente os benefícios sociais trazidos pelo projeto} \\\hline
     \fontsize{10}{12}\selectfont 
     
     
     TEXTO
     
     
     \\ \hline
\end{longtable}
\end{table}
\newpage


%===================7 - Produção tecnológica===================%
\section{Avaliação da gestão do projeto}
\vspace{-0.8cm} % Espaço opcional entre a linha
\rule{\textwidth}{2pt} % Linha preenchendo a largura do texto e com 2pt de espessura
\vspace{-0.9cm} % Subir tabela
\begin{table}[h]
    \begin{longtable}{|p{17.4cm}|} % Definindo a largura da coluna
     \hline
     \rowcolor{lightgray} 
     \fontsize{8}{10}\selectfont{Apresente o(s) produto(s), protótipo(s), patente(s), processo(s), metodologia(s) que surgiram em meio ao projeto e mostraram inovação e relevância, mas que não haviam sido previstos como indicadores físicos.}  
     \\ \hline \endhead
     
     \fontsize{10}{12}\selectfont 
     
     
     TEXTO
     
     
     
     \\ \hline
     \end{longtable}
     \end{table}



%===================8 - Parceria institucional===================%
\section{Parceria institucional}
\vspace{-0.8cm} % Espaço opcional entre a linha
\rule{\textwidth}{2pt} % Linha preenchendo a largura do texto e com 2pt de espessura
\vspace{-0.9cm} % Subir tabela
\begin{table}[h]
    \begin{longtable}{|p{17.4cm}|} % Definindo a largura da coluna
     \hline
     \rowcolor{lightgray} 
     \fontsize{8}{10}\selectfont{Descreva a(s) atividade(s) de articulação institucional mantida(s) durante a execução do projeto, relacionando os resultados efetivamente transferidos para instituições de P\&D, empresas, órgãos públicos, instituições não governamentais, assim como a contribuição específica de cada instituição partícipe do Termo de Outorga.}  
     \\ \hline \endhead
     
     \fontsize{10}{12}\selectfont 
     
     
     TEXTO
     
     
     
     \\ \hline
     \end{longtable}
     \end{table}

%===================9- Comentário final===================%
\section{Comentário Final}
\vspace{-0.8cm} % Espaço opcional entre a linha
\rule{\textwidth}{2pt} % Linha preenchendo a largura do texto e com 2pt de espessura
\vspace{-0.9cm} % Subir tabela
\begin{table}[h]
    \begin{longtable}{|p{17.4cm}|} % Definindo a largura da coluna
     \hline
     \rowcolor{lightgray} 
     \fontsize{8}{10}\selectfont{Acrescente observações relevantes, que não se aplicariam aos outros campos do relatório.}  
     \\ \hline \endhead
     
     \fontsize{10}{12}\selectfont 
     
     
     TEXTO
     
     
     
     \\ \hline
     \end{longtable}
     \end{table}
\newpage
%===================10 - Resumo===================%
\section{Resumo.\protect\footnotemark[7]}
\vspace{-0.8cm} % Espaço opcional entre a linha
\footnotetext[7]{O preenchimento do resumo é obrigatório, caso se trate de relatório final de projeto.}
\rule{\textwidth}{2pt} % Linha preenchendo a largura do texto e com 2pt de espessura
\vspace{-0.9cm} % Subir tabela
\begin{table}[h]

    \begin{longtable}{|p{17.4cm}|} % Definindo a largura da coluna
     \hline
     \rowcolor{lightgray} 
     \fontsize{8}{10}\selectfont \textbf{Redija um resumo do projeto com até 200 palavras, destacando até seis palavras-chave que melhor caracterizem os resultados, que poderá ser utilizado para divulgação externa.}  
     \\ \hline \endfirsthead % Cabeçalho da primeira página
     
     \fontsize{10}{12}\selectfont 
     
     TEXTO
     
     \\ \hline
     \end{longtable}

\end{table}
\newpage
%===================11 - Equipe atual responsável pela execução técnica do projeto===================%
\section{Equipe atual responsável pela execução técnica do projeto\protect\footnote[8]{O preenchimento desta lista não exime a beneficiária proponente de solicitar formal e previamente à Finep qualquer alteração da equipe executora inicialmente aprovada para o projeto.}}
\vspace{-0.8cm} % Espaço opcional entre a linha
\rule{\textwidth}{2pt} % Linha preenchendo a largura do texto e com 2pt de espessura

\begin{table}[h]
\centering
\resizebox{\textwidth}{!}{ % Ajusta a tabela para a largura total da página
\fontsize{8}{10}\selectfont
    \begin{tabular}{|Sc|Sc|Sc|Sc|Sc|}
        \hline
        \rowcolor{lightgray}
        \makecell{Nome completo\\ \textit{(sem abreviações)}} &  
        \makecell{CPF\\ \textit{(0123456789-01)}} & 
        \makecell{Período de contratação\\ 
\textit{(de mm/aaaa até mm/aaaa)}} &
        \makecell{Fonte dos recursos para \\ pagamento do profissional \\
        \textit{(Finep ou Contrapartida)}}  &
        \makecell{Origem do componente \\ 
        \textit{(Proponente / Executor / } \\
        \textit{Interveniente)}} \\ \hline
        
        & & & & \\ \hline
        \rowcolor{Gainsboro!60} & & & & \\ \hline
        & & & & \\ \hline
        \rowcolor{Gainsboro!60} & & & & \\ \hline
        & & & & \\ \hline
            
        \end{tabular}}
\end{table}

\newpage
%===================12 - Orçamento===================%
\section{Orçamento\protect\footnotemark[9]\protect\footnotemark[10]} 
\footnotetext[9]{Planilha de preenchimento obrigatório}
\footnotetext[10]{O preenchimento desta planilha não substitui os outros formulários de prestação de contas}
\vspace{-0.8cm} % Espaço opcional entre a linha
    \rule{\textwidth}{2pt} % Linha preenchendo a largura do texto e com 2pt de espessura
    
\begin{table}[h]
\fontsize{10pt}{12pt}\selectfont
\centering
\resizebox{\textwidth}{!}{ % Ajusta a tabela para a largura total da página
\begin{tabular}{|l|l|l|l|l|l|}
    \hline
    \rowcolor{lightgray}
    %-------Cabeçalho-------%
    \textbf{ITEM} &  
    \textbf{DESCRIÇÃO} & 
    \makecell{\textbf{VALORES} \\ \textbf{LIBERADOS} \\ \textbf{PELA FINEP} \\ \textbf{(R\$)}} &
    \makecell{\textbf{VALORES} \\ \textbf{UTILIZADOS}  \\ \textbf{(R\$)}}  &
    \makecell{\textbf{SALDO} \\ \textbf{(R\$)}} &
    \textbf{(\%)} \\
    \hline

    %-------Corpo da tabela-------%
    \rowcolor{blue!40} 
    \textbf{12.1} & 
    \textbf{DESPESAS CORRENTES} & 
    \textbf{0,00} & 
    \textbf{0,00} & 
    \textbf{0,00} & 
    \textbf{0,00} \\ 
    
    \rowcolor{Gainsboro!60} 
    12.1.1 & 
    Vencimento e vantagens & 
    ADD & 
    ADD & 
    \textbf{0,00} & 
    \textbf{0,00} \\ 
    
    12.1.2 & 
    Encargos & 
    ADD & 
    ADD &
    \textbf{0,00} &
    \textbf{0,00} \\ 
    
    \rowcolor{Gainsboro!60} 
    12.1.3 & 
    Diárias(Pessoal Civil/Militar) & 
    ADD & 
    ADD & 
    \textbf{0,00} &
    \textbf{0,00} \\ 

    12.1.4 & 
    Material de Consumo & 
    ADD & 
    ADD &
    \textbf{0,00} &
    \textbf{0,00} \\ 

    \rowcolor{Gainsboro!60} 
    12.1.5 & 
    Passagens e Despesas c/Locomoção & 
    ADD & 
    ADD & 
    \textbf{0,00} &
    \textbf{0,00} \\    

    12.1.6 & 
    Serviços de consultoria & 
    ADD & 
    ADD &
    \textbf{0,00} &
    \textbf{0,00} \\ 

    \rowcolor{Gainsboro!60} 
    12.1.7 & 
    Outros Serviços de Terceiros P. Física & 
    ADD & 
    ADD & 
    \textbf{0,00} &
    \textbf{0,00} \\ \hline 

    \textcolor{blue}{12.1.8} &
    \textcolor{blue}{Outros Serviços de Terceiros P. Física} & 
    \textcolor{blue}{ADD} & 
    \textcolor{blue}{ADD} &  
    \textcolor{blue}{\textbf{0,00}} &
    \textcolor{blue}{\textbf{0,00}} \\  

    \rowcolor{Gainsboro!60} 
    12.1.8.1 & 
    Despesas Acessórios c/Importação & 
    ADD & 
    ADD & 
    \textbf{0,00} &
    \textbf{0,00} \\ 

    12.1.8.2 & 
    Outras Despesas & 
    ADD & 
    ADD &
    \textbf{0,00} &
    \textbf{0,00} \\ \hline
    
    \rowcolor{blue!40} \textbf{12.2} & 
    \textbf{DESPESAS DE CAPITAL} & 
    \textbf{0,00} & 
    \textbf{0,00} & 
    \textbf{0,00} & 
    \textbf{0,00} \\ \hline

    \rowcolor{Gainsboro!60} 
    \textcolor{blue}{12.2.1} &
    \textcolor{blue}{Obras e Instalações} & 
    \textcolor{blue}{ADD} & 
    \textcolor{blue}{ADD} &  
    \textcolor{blue}{\textbf{0,00}} &
    \textcolor{blue}{\textbf{0,00}} \\  

    12.2.1.1 & 
    Obras & 
    ADD & 
    ADD &
    \textbf{0,00} &
    \textbf{0,00} \\ 

    \rowcolor{Gainsboro!60}
    12.2.1.2 & 
    Instalações & 
    ADD & 
    ADD &
    \textbf{0,00} &
    \textbf{0,00} \\ \hline

    \textcolor{blue}{12.2.2} &
    \textcolor{blue}{Equipamento e Material Permanente} & 
    \textcolor{blue}{ADD} & 
    \textcolor{blue}{ADD} &  
    \textcolor{blue}{\textbf{0,00}} &
    \textcolor{blue}{\textbf{0,00}} \\  

    12.2.2.1 & 
    Equipamento Nacional & 
    ADD & 
    ADD &
    \textbf{0,00} &
    \textbf{0,00} \\ 
       
    \rowcolor{Gainsboro!60}
    12.2.2.2 & 
    Equipamento Importado & 
    ADD & 
    ADD &
    \textbf{0,00} &
    \textbf{0,00} \\ 

    12.2.2.3 & 
    Material Permanente Nacional & 
    ADD & 
    ADD &
    \textbf{0,00} &
    \textbf{0,00} \\ 

    \rowcolor{Gainsboro!60}
    12.2.2.4 & 
    Material Permanente Importado & 
    ADD & 
    ADD &
    \textbf{0,00} &
    \textbf{0,00} \\ \hline

    \rowcolor{blue!40} \textbf{12.3} & 
    \textbf{TOTAIS} & 
    \textbf{0,00} & 
    \textbf{0,00} & 
    \textbf{0,00} & 
    \textbf{0,00} \\ \hline\hline

    \rowcolor{blue!40} \textbf{12.4} & 
    \textbf{APLICAÇÕES FINANCEIRAS} & 
    \multicolumn{3}{|l||}{\textbf{0,00}} & 
    \textbf{0,00} \\ \hline
    
\end{tabular}}
\end{table}


\vspace{1cm}

\centering
[LOCAL,DATA]

\vspace{1cm}
\underline{\hspace{10cm}} \\[1em] % Linha de assinatura 
[Nome completo do coordenador do projeto] [Título/cargo na instituição]
\newpage

%===================13 - Instruções adicionais para preenchimento do relatório (1ª parte)===================%
\section{Instruções adicionais para preenchimento do relatório (1ª parte)}
\vspace{-0.8cm} % Espaço opcional entre a linha
    \rule{\textwidth}{2pt} % Linha preenchendo a largura do texto e com 2pt de espessura

\raggedright Instrução de preenchimento da tabela resumo e da execução do cronograma físico do projeto

Exemplo de cronograma de projeto
\\[0.5cm] % Espaço entre as tabelas
\textbf{ADD IMAGEM CRONOGRAMA}
\\[0.5cm] % Espaço entre as tabelas
Preenchimento da tabela resumo (item 3), baseado no exemplo de cronograma de projeto
\\[0.5cm] % Espaço entre as tabelas
Relatório do 1º período

\begin{table}[h]
\centering
\fontsize{12}{10}\selectfont
\resizebox{\textwidth}{!}{ % Ajusta a tabela para a largura total da página
\begin{tabular}{|p{2cm}|p{3cm}|p{4cm}|p{3cm}|p{2cm}|p{2cm}|p{2cm}|p{2cm}|p{2cm}|p{2cm}|p{2cm}|p{2cm}|}
    \hline
    \hline
    \rowcolor{lightgray}
    \textbf{Item Meta}&
    \textbf{Metas}&
    \textbf{Atividades}&
    \textbf{Indicador \newline Físico}&
    \multicolumn{2}{c|}{\textbf{Duração Prevista}} &
    \multicolumn{2}{c|}{\textbf{Duração efetiva}} &
    \multicolumn{2}{c|}{\textbf{Executado no período}} &
    \multicolumn{2}{c|}{\parbox[t]{3cm}{\textbf{Acumulado da meta atividade}}}  \\ \cline{5-12}
    
    \rowcolor{lightgray}
    & & & & 
    \textbf{Mês/Ano inicio} & \textbf{Mês/Ano fim} &
    \textbf{Mês/Ano inicio} & \textbf{Mês/Ano fim} &
    \textbf{Previsto (\%)} & \textbf{Realizado (\%)} & \textbf{Previsto (\%)} & \textbf{Realizado (\%)} 
    \\ \hline


    %PREENCHER [1]
    [1] & 
    Texto descrevendo a meta física 1 
    &
    &
    & 01/10 
    & 12/10 
    & 02/10 
    & - & 
    6m/12m = 50\% & 
    4m/12m = 33\% & 
    6m/12m = 50\% &
    4m/12m = 33\% \\
    \hline

   %PREENCHER [1.1]
    [1.1] & 
    Texto descrevendo a meta física 1 
    &
    &
    & 01/10 
    & 12/10 
    & 02/10 
    & - & 
    6m/12m = 50\% & 
    4m/12m = 33\% & 
    6m/12m = 50\% &
    4m/12m = 33\% \\
    \hline 
\end{tabular}}\end{table}
 

Relatório do 1° período. 
% META FÍSICA 1
\begin{table}[h]
\fontsize{8}{10}\selectfont
\centering
\resizebox{\textwidth}{!}{ % Ajusta a tabela para a largura total da página
\begin{tabular}{| p{9cm} | p{1.5cm} | p{1.5cm} | p{1.5cm} | p{1.5cm} |}
    \hline
    \rowcolor{lightgray}
    \multirow{2}{*}{\textbf{Meta física 1}}&
    \multicolumn{2}{c|}{\textbf{Executado no período}} &
    \multicolumn{2}{c|}{\parbox[t]{3cm}{\textbf{Acumulado da meta \newline durante o projeto}}}  \\ \cline{2-5}
    
    \rowcolor{lightgray}
    & \textbf{Previsto (\%)} & \textbf{Realizado (\%)} & \textbf{Previsto (\%)} & \textbf{Realizado (\%)} \\ \hline

    [Texto descrevendo a meta física] & 50 & 33 & 50 & 33 \\ \hline

\end{tabular}}
\\[0.5cm] % Espaço entre as tabelas


% ATIVIDADE 1.1
% Ajusta a tabela para a largura total da página 

\resizebox{\textwidth}{!}{
\begin{tabular}{| p{4.05cm} | p{4.5cm} | p{1.5cm} | p{1.5cm} | p{1.5cm} | p{1.5cm} |}
    \hline
    \rowcolor{lightgray}
    \multirow{2}{*}{\textbf{Atividade 1.1}} & 
    \multirow{2}{*}{\textbf{Indicador físico de execução}} & 
    \multicolumn{2}{c|}{\textbf{Executado no período}} &
    \multicolumn{2}{c|}{\parbox[t]{3cm}{\textbf{Acumulado da meta \newline durante o projeto}}}  \\ \cline{3-6}
    
    \rowcolor{lightgray}
    & & \textbf{Previsto (\%)} & \textbf{Realizado (\%)} & \textbf{Previsto (\%)} & \textbf{Realizado (\%)} \\ \hline

    [Texto descrevendo a atividade] & [Texto descrevendo o indicador físico] & 75 & 50 & 75 & 50 \\ \hline
\end{tabular}}
\\[0.5cm] % Espaço entre as tabelas

% ATIVIDADE 1.2
\resizebox{\textwidth}{!}{ % Ajusta a tabela para a largura total da página
\begin{tabular}{| p{4.05cm} | p{4.5cm} | p{1.5cm} | p{1.5cm} | p{1.5cm} | p{1.5cm} |}
    \hline
    \rowcolor{lightgray}
    \multirow{2}{*}{\textbf{Atividade 1.1}} & 
    \multirow{2}{*}{\textbf{Indicador físico de execução}} & 
    \multicolumn{2}{c|}{\textbf{Executado no período}} &
    \multicolumn{2}{c|}{\parbox[t]{3cm}{\textbf{Acumulado da meta \newline durante o projeto}}}  \\ \cline{3-6}
    
    \rowcolor{lightgray}
    & & \textbf{Previsto (\%)} & \textbf{Realizado (\%)} & \textbf{Previsto (\%)} & \textbf{Realizado (\%)} \\ \hline

    [Texto descrevendo a atividade] & [Texto descrevendo o indicador físico] & 75 & 50 & 75 & 50 \\ \hline
\end{tabular}}
\end{table}
\newpage

%===================14 - Instruções adicionais para preenchimento do relatório (2ª parte)===================%
\section{Instruções adicionais para preenchimento do relatório (2ª parte)}
\vspace{-0.8cm} % Espaço opcional entre a linha
    \rule{\textwidth}{2pt} % Linha preenchendo a largura do texto e com 2pt de espessura
    
% Definindo o estilo da caixa retangular
\begin{mdframed}[linecolor=black,linewidth=1pt,backgroundcolor=gray!40]

\begin{enumerate}[label=14.\arabic*.,leftmargin=1cm]
\fontsize{10}{10}\selectfont
    \item O relatório deve sempre começar por um \textbf{Sumário}, para que seja facilitada a consulta de seus itens.
    \item Nenhuma das partes deste modelo de relatório deve ser suprimida pela beneficiária proponente. Quando, por qualquer razão, não houver necessidade de preenchimento de determinado item (exceto o de número 2, \textit{Informações opcionais e subitens}), o campo, com a expressão \textit{``Deixado propositalmente em branco''}, deve ser colocado.
    \item Em regra, os campos descritivos deste modelo de relatório não têm um número máximo de caracteres e, portanto, devem estender-se tanto quanto necessário. Principalmente, os campos com a descrição do desenvolvimento das atividades, que fazem parte do item \textbf{Execução do Cronograma Físico do Projeto} (item 2 do relatório). Uma das exceções sobre o limite de tamanho é o campo \textbf{Resumo}, que deve ter no máximo 200 palavras.
    \item O preenchimento da tabela resumo da execução do cronograma físico do projeto deve informar a situação atual do projeto, descrevendo as metas e atividades aprovadas, assim como as previstas e executadas no período e acumuladas.
    \item Em relação às informações sobre a \textbf{Execução do Cronograma Físico}, cada atividade de cada meta física deve ser mencionada no relatório, mesmo aquelas não iniciadas. Quando se tratar de atividade não iniciada, o campo \textbf{Execução (\%)} deverá ser preenchido com 0\% e, se a atividade estiver atrasada em relação à previsão inicial, a justificativa pelo atraso deverá ser necessariamente apresentada no campo correspondente. Caso a atividade tenha sido iniciada, mas não finalizada, o campo \textbf{Execução} deverá ser preenchido com percentual correspondente ao estágio de desenvolvimento até aquele momento.
    \item As informações sobre a execução das atividades do \textbf{Cronograma Físico}, e respectivos anexos contendo indicadores físicos, devem ser incrementais. Ou seja, mesmo que uma determinada atividade já tenha sido concluída (\textbf{``Execução'' 100\%}), as páginas com estes indicadores devem constar dos relatórios seguintes, de maneira que o relatório final contenha toda a história do projeto desde o início de sua execução.
    \item Sempre que houver um \textbf{indicador físico do projeto} concluído, a comprovação de sua execução, seja ela descritiva, gráfica, fotográfica ou por qualquer outro documento demonstrativo cabível, deverá constar em anexo ao relatório. Também deverá ser anexado ao relatório o indicador físico ligado ao estágio de execução justifique a apresentação.
    \item Sempre que indicado, e quando o estágio em que se encontra o projeto justificar, deverão ser anexados ao corpo de um dos anexos ao relatório: plantas industriais e de engenharia; figuras; gráficos; diagramas de circuito; protótipos; provas de conceito; resultados de análises laboratoriais; manuais de operação; fotografias de partes e peças mecânicas; listas de pessoal; listas de assinaturas; estudos de viabilidade; relatórios de impacto ambiental; e qualquer outro documento ou artefato que comprove a execução das atividades pactuadas.
    \item O preenchimento da planilha com as informações sobre o \textbf{orçamento do projeto} é obrigatório.
    \item A planilha com as informações sobre os valores empregados no projeto deve refletir a situação atual do projeto.
    \item Todas as páginas do relatório e dos anexos devem receber \textbf{rubricas do coordenador do projeto}, exceto feita à última página do relatório, que deve conter a \textbf{assinatura do coordenador}.
    \item Finalmente, é importante lembrar que este relatório deverá ser enviado à \textbf{Finep} pela beneficiária proponente exclusivamente por meio digital, encaminhando ao endereço eletrônico \texttt{cp\_protocolo@finep.gov.br}, com cópia por e-mail para os analistas operacionais responsáveis pelo acompanhamento do projeto.
\end{enumerate}
\end{mdframed}
\newpage

%=================== 15.ANEXO A –[preencher com o texto descrevendo o indicador físico da atividade 1.1]===================%
\section{ANEXO A – \color{blue}{[preencher com o texto descrevendo o indicador físico da atividade 1.1]}}
\vspace{-0.8cm} % Espaço opcional entre a linha
    \rule{\textwidth}{2pt} % Linha preenchendo a largura do texto e com 2pt de espessura
    
\newpage

%=================== 16.ANEXO B –[preencher com o texto descrevendo o indicador físico da atividade 1.n]===================%
\section{ANEXO B – \color{blue}{[preencher com o texto descrevendo o indicador físico da atividade 1.n]}}
\vspace{-0.8cm} % Espaço opcional entre a linha
    \rule{\textwidth}{2pt} % Linha preenchendo a largura do texto e com 2pt de espessura

\newpage

%=================== 17.ANEXO C –[preencher com o texto descrevendo o indicador físico da atividade 1.n]===================%
\section{ANEXO C – \color{blue}{[preencher com o texto descrevendo o indicador físico da atividade N.1]}}
\vspace{-0.8cm} % Espaço opcional entre a linha
    \rule{\textwidth}{2pt} % Linha preenchendo a largura do texto e com 2pt de espessura

\newpage

%=================== 18.ANEXO D –[preencher com o texto descrevendo o indicador físico da atividade N.n]===================%
\section{ANEXO D – \color{blue}{[preencher com o texto descrevendo o indicador físico da atividade N.n]}}
\vspace{-0.8cm} % Espaço opcional entre a linha
    \rule{\textwidth}{2pt} % Linha preenchendo a largura do texto e com 2pt de espessura

\end{document}