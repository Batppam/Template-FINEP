\section{Instruções adicionais para preenchimento do relatório (2ª parte)}
\vspace{-0.8cm} % Espaço opcional entre a linha
    \rule{\textwidth}{2pt} % Linha preenchendo a largura do texto e com 2pt de espessura
    
% Definindo o estilo da caixa retangular
\begin{mdframed}[linecolor=black,linewidth=1pt,backgroundcolor=gray!40]

\begin{enumerate}[label=14.\arabic*.,leftmargin=1cm]
\fontsize{10}{10}\selectfont
    \item O relatório deve sempre começar por um \textbf{Sumário}, para que seja facilitada a consulta de seus itens.
    \item Nenhuma das partes deste modelo de relatório deve ser suprimida pela beneficiária proponente. Quando, por qualquer razão, não houver necessidade de preenchimento de determinado item (exceto o de número 2, \textit{Informações opcionais e subitens}), o campo, com a expressão \textit{``Deixado propositalmente em branco''}, deve ser colocado.
    \item Em regra, os campos descritivos deste modelo de relatório não têm um número máximo de caracteres e, portanto, devem estender-se tanto quanto necessário. Principalmente, os campos com a descrição do desenvolvimento das atividades, que fazem parte do item \textbf{Execução do Cronograma Físico do Projeto} (item 2 do relatório). Uma das exceções sobre o limite de tamanho é o campo \textbf{Resumo}, que deve ter no máximo 200 palavras.
    \item O preenchimento da tabela resumo da execução do cronograma físico do projeto deve informar a situação atual do projeto, descrevendo as metas e atividades aprovadas, assim como as previstas e executadas no período e acumuladas.
    \item Em relação às informações sobre a \textbf{Execução do Cronograma Físico}, cada atividade de cada meta física deve ser mencionada no relatório, mesmo aquelas não iniciadas. Quando se tratar de atividade não iniciada, o campo \textbf{Execução (\%)} deverá ser preenchido com 0\% e, se a atividade estiver atrasada em relação à previsão inicial, a justificativa pelo atraso deverá ser necessariamente apresentada no campo correspondente. Caso a atividade tenha sido iniciada, mas não finalizada, o campo \textbf{Execução} deverá ser preenchido com percentual correspondente ao estágio de desenvolvimento até aquele momento.
    \item As informações sobre a execução das atividades do \textbf{Cronograma Físico}, e respectivos anexos contendo indicadores físicos, devem ser incrementais. Ou seja, mesmo que uma determinada atividade já tenha sido concluída (\textbf{``Execução'' 100\%}), as páginas com estes indicadores devem constar dos relatórios seguintes, de maneira que o relatório final contenha toda a história do projeto desde o início de sua execução.
    \item Sempre que houver um \textbf{indicador físico do projeto} concluído, a comprovação de sua execução, seja ela descritiva, gráfica, fotográfica ou por qualquer outro documento demonstrativo cabível, deverá constar em anexo ao relatório. Também deverá ser anexado ao relatório o indicador físico ligado ao estágio de execução justifique a apresentação.
    \item Sempre que indicado, e quando o estágio em que se encontra o projeto justificar, deverão ser anexados ao corpo de um dos anexos ao relatório: plantas industriais e de engenharia; figuras; gráficos; diagramas de circuito; protótipos; provas de conceito; resultados de análises laboratoriais; manuais de operação; fotografias de partes e peças mecânicas; listas de pessoal; listas de assinaturas; estudos de viabilidade; relatórios de impacto ambiental; e qualquer outro documento ou artefato que comprove a execução das atividades pactuadas.
    \item O preenchimento da planilha com as informações sobre o \textbf{orçamento do projeto} é obrigatório.
    \item A planilha com as informações sobre os valores empregados no projeto deve refletir a situação atual do projeto.
    \item Todas as páginas do relatório e dos anexos devem receber \textbf{rubricas do coordenador do projeto}, exceto feita à última página do relatório, que deve conter a \textbf{assinatura do coordenador}.
    \item Finalmente, é importante lembrar que este relatório deverá ser enviado à \textbf{Finep} pela beneficiária proponente exclusivamente por meio digital, encaminhando ao endereço eletrônico \texttt{cp\_protocolo@finep.gov.br}, com cópia por e-mail para os analistas operacionais responsáveis pelo acompanhamento do projeto.
\end{enumerate}
\end{mdframed}
